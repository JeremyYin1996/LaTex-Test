%导言区
\documentclass[12pt]{article}

\usepackage{ctex}

\newcommand{\myfont}{\textbf\textsf{{}}

%正文区(文稿区)
\begin{document}
	%字体族设置(罗马字体、无衬线字体、打字机字体)
	\textrm{Roman Family} \textsf{Sans Serif Family} \texttt{Typewriter Family}  
	
	\rmfamily Roman Family {\sffamily Sans Serif Family} {\ttfamily Typewriter Family}
	
	%字体系列设置(粗细、宽度)
	\textmd{细的 Medium Series}
	\textbf{粗的 Boldface Series}
	
	%字体形状(直立、斜体、伪斜体、小型大写)
	\textup{Upright Shape}
	\textit{Italic Shape}
	\textsl{Slanted Shape}
	\textsc{Small Caps Shaps}
	
	%中文字体 %quad表示空格
	{\songti 宋体} \quad {\heiti 黑体} \quad {\fangsong 仿宋} \quad {\kaishu 楷书}
	
	中文字体的\textbf{粗体}与\textit{斜体}
	
	%字体大小
   	{\tiny Hello}\\
	{\scriptsize  Hello}\\
	{\footnotesize Hello}\\
	{\small Hello}\\
	{\normalsize Hello}\\
	{\large Hello}\\
	{\LARGE Hello}\\
	{\huge Hello}\\
	{\Huge Hello}\\
	
	%中文字号设置
	\zihao{5} 你好!
	
	\myfont 这是示例文字
	
		
\end{document}