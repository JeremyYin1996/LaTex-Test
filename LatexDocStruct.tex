\documentclass[UTF8]{ctexart}

\usepackage{fancyhdr}%添加页眉页脚的宏包
\pagestyle{plain} %没有页眉,页脚中部放置页码

\usepackage[fleqn]{amsmath}%全局设置多公式对齐

\title{\heiti 多种方法求解$lnx$} %标题

\author{\songti 殷昊南} %作者

\date{\today}

%正文区(文稿区)
\begin{document}
	\maketitle %显示文章标题
	\thispagestyle{empty} %第一页的页眉页脚清空
	\begin{abstract} %摘要
	本文研究在为计算器按键设计对数键$lnx$时,对$lnx$的求法进行了分析,通过采用数值积分和泰勒展开的方法,用多项式无限接近于$lnx$的真实值,并且分析采用这两种方法后误差的产生原因和范围,计算结果精确到10位以后。
	\newline%另起一行
	\centering%使得关键字居中
	\textbf{关键字:}数值积分、泰勒公式、复化辛卜生公式、复化梯形公式
	\end{abstract}

	\newpage%另起一页	
	\tableofcontents %—— 制作目录
	\thispagestyle{empty} %第二页的页眉页脚清空
	
	\newpage%另起一页
	\setcounter{page}{1}%从此页面开始编页码
	\section{问题重述} %1
	\subsection{$lnx$的不同求法} %1.1
	\subsubsection{问题1} %1.1.1
	\par{至少给出两种方法为计算器设计对数键$lnx$,使得当任意输入一个正实数$x$时能返回其函数值$lnx$。要求给出方法的理论推导,分析方法的误差,写出算法,并编写程序。}
	\subsubsection{问题2} %1.1.2
	\par{为了得到足够准确的数字来填满计算器的数字显示屏,若要求输出结果精确到10位小数。结合你的计算方法,通过理论分析说明是否能达到该要求,并编程计算进行验证。}
	\section{问题分析} %2
		\subsection{问题1的分析} %2.1
		\par{问题1中要求给出$lnx$的两种求法,于是我们想到了数值积分和泰勒展开这两种方法,并加以计算。}
			\subsubsection{数值积分} %2.1.1
	当输入一个$a$求$lna$时,通过计算$\frac{1}{x}$在(1,$a$)上的积分值来得到
	$$	\int_1 ^a \frac{1}{x} dx = lna -ln1 =a $$
	其中积分的方法可以用复化辛普森公式和复化梯形公式进行解决。
			\subsubsection{泰勒公式展开} %2.1.2
	根据泰勒公式可以使用多项式代替$lna$,由于在使用泰勒公式:
		\begin{equation}
			\label{taile}%添加公式标签,方便引用
			ln(x+1)= \sum_{i=1}^{n} (-1)^{i+1} \frac{x^i}{i}+R_n(x),x\in(-1,1)          
		\end{equation}	
	时,$a=1+x$的取值范围要求在(0,2)之间,因为要对任意的正实数$a$进行泰勒展开,即当$a\ge2$时,通过对数函数的性质,令$lna=ln(\frac{a}{e^n})+n$,并且算$\frac{x}{e^n}$的范围在(0,2) 之间可以进行泰勒展开。
	
		\subsection{问题2的分析} %2.2
		在问题2中要求输出结果精确到10位小数,在复化梯形和复化辛普森方法中可以直接求出截断误差,在泰勒展开的方法中采用对计算出的结果和真实值直接的比较进行误差分析,来达到精度要求。
	
	\section{模型假设和符号说明} %3
		\subsection{模型假设} %3.1
		将MATLAB软件中的$log(a)$命令作为$lna$的真实值进行比较和误差分析。
		\subsection{符号说明} %3.2
	在区间$[a,b]$上将求积区间做$n$等分,并记 $h=\frac{b-a}{n}$(区间步长),$x_k = a+kh (0\leq k\leq h)$ 于是
	$$	I(f)=\sum_{k=0}^{n-1}\int_{x_k}^{x_{k+1}}f(x)dx $$
	并记
	$$	I_{k}(f)=\int_{x_k}^{x_{k+1}}f(x)dx $$
	
	\begin{table}[!htbp]  %  !-忽略“美学”标准	h-here 	t-top b-bottom      p-page-of-its-own
		\centering
		\caption{符号说明}%添加标题
		\label{tab:FuHuHaoShuoMing} %设置标签
	\begin{tabular}{cc}%一个c表示有一列,格式为居中显示(center)
		\hline  % 横线
		符号&符号说明\\
		\hline  
		$I(f)$	&所求函数的积分\\
		
		$I_k(f)$	&所求函数的积分划分成$n$等分后的子积分
\\
		
		$T_n(f)$	&复化梯形公式\\
		
		$S_n(f)$	&复化辛卜森公式\\
		
		$R_n(f)_1$	&复化梯形公式截断误差\\
		
		$R_n(f)_2$	&复化辛卜生公式截断误差\\
		
		${U_n}$	&交错级数中的数列项\\
		
		$R_n$	&交错级数的余项估计\\
		
		$h$	&区间步长\\
		\hline
	\end{tabular}	
	\end{table}	
	
	\section{模型建立与求解} %4
		\subsection{问题1的模型建立与求解} %4.1
			\subsubsection{数值积分方法} %4.1.1
	根据微积分学基本定理,若被积函数$f(x)=\frac{1}{x}$在区间$[a,b]$上连续,只要能找到$f(x)$的一个原函数$F(x)$,便可以利用牛顿-莱布尼兹公式
	$$\int_{a}^{b}f(x)dx=F(b)-F(a)$$
	即在区间$[a,1]$上 
	$$\int_{a}^{1}\frac{1}{x}dx=ln1-lna$$
	求得的积分值。
	另外积分方法采用数值积分中的复化梯形公式和复化辛卜生公式
	复化梯形公式:(具体代码参见附录1-复化梯形公式)
	\begin{equation}
		\begin{split}
		\int_{a}^{b}f(x)dx \approx T_n(f)& = \sum_{k=0}^{n-1}\frac{h}{2}[f(x_k)+f(x_{k+1})]\\
		&=\frac{h}{2}[f(x_0)+2\sum_{k=1}^{n-1}f(x_k)+f(x_n)]
		\end{split}
	\end{equation}
	复化辛卜生公式:(具体代码参见附录1-复化辛卜生公式)
	\begin{equation}
		\begin{split}
		\int_{a}^{b}f(x)dx \approx S_n(f)& = \sum_{k=0}^{n-1}\frac{h}{6}[f(x_k)+4f(x_{k+\frac{1}{2})})+f(x_{k+1})]\\
		&=\frac{h}{6}[f(x_0)+2\sum_{k=1}^{n-1}f(x_i)+4\sum_{k=0}^{n-1}f(x_{k+{\frac{1}{2}}}) +f(x_n)]
		\end{split}
	\end{equation}
			\subsubsection{泰勒公式} %4.1.2
	根据泰勒公式可以使用多项式代替$lna$,由于在使用公式(\ref{taile})
	时,$a=1+x$的取值范围要求在(0,2)之间,所以当$a \geq 2$时对原来的$lna$进行转换,通过对数函数的性质,令$lna=ln(\frac{a}{e^n})+n$,并且$\frac{a}{e^n}$的范围必定在(0,2)内,因此当$a \geq 2$时可以转化为求$ln(\frac{a}{e^n})$的泰勒展开式的公式(\ref{taile})中去。
	\subsection{问题2的模型建立与求解}%4.2
	\subsubsection{数值分析方法的误差分析}%4.2.1
	\par{复化梯形公式的截断误差为}
	$$   R_n(f)_1=I(f)-T_n(f)=\sum_{k=1}^{n}-\frac{h^3}{12}f^{''}(\eta)=-\frac{b-a}{2}f^{''}(\eta)h^2  ,\eta \in(a,b) $$
	\par{复化辛卜生公式的截断误差为}
	$$   R_n(f)_2=I(f)-S_n(f)=\sum_{k=1}^{n}-\frac{h^5}{2880}f^{(4)}(\eta_k)=-\frac{b-a}{2880}f^{(4)}(\eta)h^4  ,\eta \in(a,b) $$
	\par{在使用复化梯形公式和复化辛卜生公式时,会发现在要求相同精度的条件下,得到同一个数值时,复化辛卜生方法分割的区间更少,收敛速度更快。}
	\par{根据上面的两个公式我们可以知道,复化梯形公式是2阶收敛的,复化辛卜生公式是4阶收敛的,在使用复化梯形公式和复化辛卜生公式计算$lna$的值且误差精度达到$\frac{1}{2}\times 10^{-11}$时的运算结果和分割次数如下表\ref{tab:FenGeCiShu}所示:}
	\begin{table}[!htbp]
		\centering
		\caption{复化梯形公式和复化辛卜生公式计算结果和分割次数}%添加标题
		\label{tab:FenGeCiShu} %设置标签
		\begin{tabular}{c|cc|cc}
			\hline 
			$a$   & $T_n(f)$         & 对分次数  & $S_n(f)$      & 对分次数 \\ 
			\hline 
			0.01  & -4.60517018599   & $2^{24}$ & -4.60517018599 &  $2^{14}$ \\ 
			0.1   & -2.30258509300   & $2^{21}$ & -2.30258509299 &  $2^{11}$ \\ 
			10    & 2.30258509300    & $2^{21}$ & 2.30258509299  &  $2^{11}$ \\ 
			20    & 2.99573227356    & $2^{22}$ & 2.99573227355  &  $2^{12}$ \\ 
			40    & 3.68887945414    & $2^{23}$ & 3.68887945412  &  $2^{13}$ \\ 
			80    & 4.38202663470    & $2^{24}$ & 4.38202663467  &  $2^{14}$ \\
			100   & 4.60517018599    & $2^{24}$ & 4.60517018599  &  $2^{14}$ \\ 
			\hline 
		\end{tabular} 
	\end{table}	
	\par{ 由表\ref{tab:FenGeCiShu}可知在使用数值积分的方法求$lna$时,使用复化辛卜生公式更好,并且这两种方法的输出结果都可以精确到10位小数以上。}
	
	\subsubsection{泰勒公式的误差分析} %4.2.2
	\par{当输入的$a$的取值范围在(0,2)时,误差来源于对$ln(1+x)$泰勒展开式的误差,
	而当$a\geq2$ 时,由于已经将 $lna=ln(\frac{a}{e^n})+n $,所以在求$lna$时的误差就转变成 $ln(\frac{a}{e^n})$泰勒展开时产生的误差和$\frac{a}{e^n}$ 的舍入误差。}
	\par{在使用泰勒公式进行计算时,我们采用的是比较输出结果和MATLAB自带命令 之差的绝对值来达到输出结果的精度要求。}
	\par{另外由于泰勒级数展开式的通项是交错级数,根据交错级数的性质,其余项估计为$R_n\leq U_{n+1}$ ,但是经测试后发现:虽然实验结果满足题目要求,但是会出现泰勒级数的估计值小于截断误差,如:在使用泰勒级数计算$ln0.01$时,计算的截断误差为$0.4782574\times10^{-10}$,而其余项值为$0.4950324\times10^{-10}$,对此,我们的推论是由于$log(a)$命令自身产生的舍入误差所导致。}
	
	\subsubsection{两种方法的比较与分析} %4.2.3
	\par{在上面已经比较过数值积分的两种方法中复化辛卜生法分割次数更少,运算时间更短,当复化辛卜生和泰勒展开法进行比较结果如下表\ref{tab:BiJiaoShiJian}:}
		\begin{table}[!htbp]
		\centering
		\caption{复化辛卜生公式和泰勒展开法计算结果精确程度和运行时间}%添加标题
		\label{tab:BiJiaoShiJian} %设置标签
		\begin{tabular}{ccccc}
			\hline 
			$a$ & 复化辛卜生 & 复化辛卜生所用时间(运算10次后的平均值) & 泰勒公式 & 泰勒公式所用时间(运算10次后的平均值) \\ 
			\hline 
			10 & 2.30258509299 & 0.042054秒 & 2.30258509299 & 0.007378 秒 \\ 
			100 & 4.60517018599 & 0.197414秒 & 4.60517018599 & 0.001893 秒 \\ 
			1000 & 6.90775527898 & 3.073452秒 & 6.90775527899 & 0.000762 秒 \\ 
			10000 & 9.21034037198 & 24.672889秒 & 9.21034037198 & 0.000881 秒 \\ 
			\hline 
		\end{tabular} 
	\end{table}	
	\par{根据表\ref{tab:BiJiaoShiJian}可知,在求相同数值,且结果精确要求相同时,使用泰勒公式的运算时间更少,运算速度更快。	}
	\section{模型的评价与改进} %5
		\subsection{模型的评价} %5.1
			\subsubsection{模型的优点} %5.1.1
	\par{该模型运用了多种方法求出了$lnx$的值,并且计算结果达到了题目所要求的精度要求。}
			\subsubsection{模型的缺点} %5.1.2
	\par{(1)该模型是直接使用的MATLAB软件中的$log(a)$命令来求出真实值后再进行比较的,使用$log(a)$命令时其自身也有可能会产生舍入误差,另外泰勒公式的截断误差的判断根据$log(a)$命令所得真实值和所求值之差的绝对值达到了精度要求,因此有可能使误差进一步放大。}
	\par{(2)该模型只能人工设定误差值范围后再进行计算。}
	\par{(3)该模型的测试值能局限于所得的$lnx$的值在$(0,99)$之内,即输入的$x$的值在$(0,e^{99})$之中保证可以达到进度要求,但是在该范围之外还未进行计算。}
		\subsection{模型的改进} %5.2
		\par{在使用泰勒公式对 $lnx$进行展开时还可以采用下面这种方法
			$$ ln(x)=ln(\frac{1+y}{1-y})= 2y(\frac{1}{1}+\frac{1}{3}y^2+\frac{1}{5}y^4    +\frac{1}{7}y^6+\frac{1}{9}y^8+... )$$
		其中$y=\frac{x-1}{x+1} $,这种方法的泰勒级数展开式收敛的更快,但由于时间不充裕,无法将这种方法进行实现。}

	%参考文献(一次性使用)
	\begin{thebibliography}{99}
		\addcontentsline{toc}{section}{参考文献}%将“参考文献加入目录中”
		\bibitem{book1} Gerald Recktenwald.\emph{《数值方法和MATLAB实现与应用》}[M].北京:机械工业出版社,2004.
		\bibitem{book2} 姜键飞.\emph{《数值分析及其MATLAB实验》}[M].北京:清华大学出版社,2004.
		\bibitem{book3} 孙志忠.\emph{《计算方法与实习》}[M].南京.东南大学出版社,2011.
		\bibitem{article1} 张春红.\emph{泰勒公式在近似计算及数值积分中的应用}[J].黑龙江科学,2014,(7):136-137
		\bibitem{book5} 李庆扬.《数值分析第5版》[M].北京:清华大学出版社,2008.
	\end{thebibliography}
	
	
	
	
\end{document}